\ifisGerman
    \chapter{Stand der Forschung}
\else
    \chapter{Related Work}
\fi
\label{sec:relwork}

In \autoref{sec:relwork} ist der \nameref{sec:relwork} zu finden.


\autoref{tab:passes_paperreading} shows hot to read \gls{it} papers.

\begin{table}[h]
\centering
\begin{tabular}{|c|c|}
    \hline
    \textbf{pass} & \textbf{content to read} \\ \hline
    \multirow{3}{*}{first} & title, abstract, introduction \\ \cline{2-2}
                           & section and sub-section header \\ \cline{2-2}
                           & conclusion \\ \cline{2-2}
                           & references \\ \hline
    second & content, ignoring details (proofs,...) \\ \hline
    third & content in detail (re-implement paper) \\ \hline
\end{tabular}
\caption{Passes their content for reading a paper}
\label{tab:passes_paperreading}
\end{table}


In \autoref{fig:fhlogo} kann das FH-Logo bestaunt werden.

\begin{figure}[ht]
    \centering
    \includegraphics[scale=0.2]{gfx/fh_logo}
    \caption{FH-Logo}
    \label{fig:fhlogo}
\end{figure}


Bilder können auch mit PGF und TikZ gezeichnet werden.
Beispiele hierfür sind \autoref{fig:paperreading} und \autoref{fig:flow_mwissa}.

\begin{figure}[ht]
    \centering
    % First pass of the paper reading process
\smartdiagramset{
    back arrow disabled=true,
    module minimum width=5cm,
    text width=5cm
}
\smartdiagram[flow diagram]{
    title,
    abstract,
    introduction,
    section and subsection headings,
    conclusion,
    references
}


    \caption{Steps of the first pass when reading a research paper}
    \label{fig:paperreading}
\end{figure}


\begin{figure}[ht]
    \centering
    % First pass of the paper reading process

\begin{tikzpicture}
\tikzset{
    mynode/.style={rectangle, rounded corners, draw=black},
    myarr/.style={->, thick}
}

\node[mynode] (learn) {Wissen aneignen};
\node[mynode, below = of learn] (choose) {Thema wählen};
\node[mynode, right= of choose] (suggest) {Thema vorschlagen};
\node[mynode, below = of choose] (research) {Analyse des Themenbereichs};
\node[mynode, below = of research] (write) {Survey/Reviewpaper schreiben};
\node[mynode, below = of write] (present) {Überblick und Forschungsfrage präsentieren};

\draw[myarr] (learn) -- (choose);
\draw[myarr] (learn) -- (suggest);
\draw[myarr] (choose) -- (research);
\draw[myarr] (suggest) -- (research);
\draw[myarr] (research) -- (write);
\draw[myarr] (write) -- (present);

\end{tikzpicture}

    \caption{Flussdiagram BAK1}
    \label{fig:flow_mwissa}
\end{figure}

Infos zu PGF/TikZ sind auf den folgenden Webseiten verfügbar:
\begin{itemize}
    \item \url{https://en.wikibooks.org/wiki/LaTeX/PGF/TikZ}
    \item \url{http://www.texample.net/tikz/examples/}
\end{itemize}




Das Halteproblem~\cite{turing_halting_prob} von \citeauthor{turing_halting_prob} kann im .bib file wie in \autoref{lstBibtex} gezeigt hinzugefügt werden und mit cite zitiert werden.



\begin{lstlisting}[style=bibtexListing,label=lstBibtex,caption={BibTeX}]
@article{turing_halting_prob,
    author = {Turing, A. M.},
    title = {On Computable Numbers, with an Application to the Entscheidungsproblem},
    journal = {Proceedings of the London Mathematical Society},
    volume = {s2-42},
    number = {1},
    publisher = {Oxford University Press},
    issn = {1460-244X},
    url = {http://dx.doi.org/10.1112/plms/s2-42.1.230},
    doi = {10.1112/plms/s2-42.1.230},
    pages = {230--265},
    year = {1937}
}
\end{lstlisting}
